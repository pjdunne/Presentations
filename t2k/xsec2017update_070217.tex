\documentclass[hyperref=colorlinks]{beamer}
\mode<presentation>
\usetheme{iclpt}
\setbeamertemplate{navigation symbols}{}
\setbeamertemplate{headline}{
  \begin{beamercolorbox}[leftskip=.2cm,rightskip=.2cm,topskip=.2cm,ht=1.1cm,dp=0.1cm,wd=\textwidth]{institute in head/foot}
    \includegraphics[height=1cm]{icl.pdf}
    \hfill
%    \includegraphics[height=1cm]{../Pics/ATLAS-Logo-Square-Blue-RGB.png}
%    \includegraphics[height=1cm]{../Pics/CMS-Color.pdf}
    \includegraphics[height=1cm]{TalkPics/t2k_logo_large.png}

%??put t2k logo here
  \end{beamercolorbox}
}
\setbeamertemplate{footline}{
  \begin{beamercolorbox}[ht=.35cm,dp=0.2cm,wd=\textwidth,leftskip=.3cm]{author in head/foot}%
    \begin{minipage}[c]{5cm}%
      \usebeamerfont{author in head/foot}
      \insertshortauthor 
      \insertshorttitle
    \end{minipage}\hfill%
    \hfill
    \insertframenumber{} / \ref{lastframe}
    %\hfill
    \begin{minipage}{6cm}
      \hfill
      %\insertshorttitle
    \end{minipage}
  \end{beamercolorbox}%
}

\definecolor{beamer@icdarkblue}{RGB}{0,51,102}
\definecolor{beamer@icmiddleblue}{RGB}{0,82,150} 
\definecolor{beamer@iclightblue}{RGB}{200,212,232}
\definecolor{beamer@icmiddlered}{RGB}{204,51,0}
\definecolor{beamer@iclightred}{RGB}{232,212,32}

\usepackage{tikz}
\usetikzlibrary{arrows,shapes,backgrounds}
\usepackage{color}
\usepackage{tabularx,colortbl}
\usepackage{graphicx}
\usepackage{pdfpages}
\usepackage{feynmp}
\usepackage{rotating}
\usepackage{moresize}
\usepackage{slashed}
\usepackage{xcolor,colortbl}
\DeclareGraphicsRule{*}{mps}{*}{}
\hypersetup{colorlinks=false}

\title[Transverse Variables for HPTPC]{\vspace{-0.2cm} Spline production and BeRPA update}
\author[P. Dunne]{Patrick Dunne - Imperial College London}
\titlegraphic{
  \vspace{-0.4cm}
}
\date{}
\begin{document}
\tikzstyle{every picture}+=[remember picture]
\tikzstyle{na} = [baseline=-.5ex]
\begin{fmffile}{t2ktemplatefeyndiags}


  %TITLE PAGE
  %20 mins + 5 questions
  \section{Title}
  \begin{frame}
    \titlepage
  \end{frame}

  \begin{frame}
    \frametitle{Overview}
    \begin{block}{}
        \begin{itemize}
        \item 2 Main packages used for spline production:
          \begin{itemize}
          \item T2KReWeight and things that go into it, e.g. NIWGReWeight
          \item XsecResponse
          \end{itemize}
        \item Need a T2KReWeight freeze before spline production can progress
        \item Kendall has gone through most of the T2KReWeight status
        \item Will cover XsecResponse status and BeRPA progress in T2KReWeight
      \end{itemize}
    \end{block}
  \end{frame}

  \begin{frame}
    \frametitle{BeRPA}
    \scriptsize
      \begin{equation*}
        f(x)=\begin{cases}
        A(1-x)^3 + 3B(1-x)^2 x + 3C(1-x)x^2 + Dx^3, q^{2}<U \\
        1+\left(D-1\right)e^{-E\left(q^{2}-U\right)}, q^{2} \geq U
        \end{cases}
      \end{equation*}
      \begin{block}{}
        \begin{itemize}
        \item BeRPA implementation for Wing's tests done by Raj and Wing using 2015 splines T2KReWeight freeze
        \item[-] Event-by-event weight stored in mtuple for nominal value
        \item[-] Binned splines made for variations about that nominal
        \item As Kendall mentioned/s there is a BeRPA implementation in the current T2KReWeight
        \item[-] It does both of the above steps in one
        \item[-] May not perform as well due to large nominal$\rightarrow$nominal+BeRPA weights
        \item 2 options: implement dial in T2KReWeight as Raj has or use current T2KReWeight dial and revalidate
        \end{itemize}
      \end{block}
  \end{frame}

  \begin{frame}
    \frametitle{Aside: BeRPA naming}
    \begin{block}{}
      \begin{itemize}
      \item In presentations BeRPA parameters that vary are ABDEU
      \item[-] C is fixed by continuity
      \item In code, names of variables are often ABCDU
      \item These map to presented BeRPA parameters as A$\rightarrow$A, B$\rightarrow$B, C$\rightarrow$D, D$\rightarrow$E, U$\rightarrow$U
      \item Awaiting TN309 for official naming
      \end{itemize}
      \end{block}
  \end{frame}

  \begin{frame}
    \frametitle{XsecResponse}
    \begin{block}{}
      \begin{itemize}
      \item Takes weights from T2KReWeight as input and makes splines
      \item Already set up to make Erec-theta, Erec and p-theta splines
      \item T2KReWeight output is a set of weights for each 'tweak' in the order they were run in the T2KReWeight executable
      \item[-] Need to know the number of variations for each dial and the order they were run in the exec
      \end{itemize}
    \end{block}
  \end{frame}

  \begin{frame}
    \frametitle{Summary}
    \label{lastframe}
    \begin{block}{}
      \begin{itemize}
      \item BeRPA implementation now understood
      \item[-] Need to decide which approach to use and put it into T2KReWeight
      \item XsecResponse work as far as it can be without some T2KReWeight details
      \item After these two are done will be ready to generate splines after T2KReWeight freeze
      \end{itemize}
    \end{block}
  \end{frame}

  %Backup goes here
  
\end{fmffile}
\end{document}

\begin{frame}
\end{frame}
