\documentclass[hyperref=colorlinks]{beamer}
\mode<presentation>
\usetheme{iclpt}
\setbeamertemplate{navigation symbols}{}
\setbeamertemplate{headline}{
\begin{beamercolorbox}[leftskip=.2cm,rightskip=.2cm,topskip=.2cm,ht=1.1cm,dp=0.1cm,wd=\textwidth]{institute in head/foot}
  \includegraphics[height=1cm]{icl.pdf}
  \hfill
  \includegraphics[height=1cm]{../Pics/CMS-Color.pdf}
\end{beamercolorbox}
}
\setbeamertemplate{footline}{
\begin{beamercolorbox}[ht=.55cm,dp=0.4cm,wd=\textwidth,leftskip=.3cm]{author in head/foot}%
  \begin{minipage}[c]{5cm}%
    \usebeamerfont{author in head/foot}
    \insertshortauthor 
    \insertshorttitle
    \end{minipage}\hfill%
  \insertframenumber{} / \pageref{lastframe}
  \hfill
  \begin{minipage}{6cm}
    \hfill
  \end{minipage}
\end{beamercolorbox}%
}

\usepackage{color}
\usepackage{tabularx,colortbl}
\usepackage{graphicx}
\usepackage{pdfpages}
\usepackage{feynmp}
\usepackage{tikz}
\usetikzlibrary{calc, shapes, backgrounds,arrows,positioning}
\DeclareGraphicsRule{*}{mps}{*}{}

\title{\vspace{-0.2cm} VBF Higgs to Invisible}
\subtitle{HIG-14-038, AN-14-243\vspace{-0.7cm}}
\author[]{}%\underline{P. Dunne}} % A.M. Magnan and A. Nikitenko Joao Pela with \\ R. Aggleton, J. Brooke: Bristol \\ C.Asawangtrakuldee, Q.Li: Peking \\ P. Srimanobhas: Chulalongkorn \\ S. Kumar, K. Mazumdar: Mumbai}
\titlegraphic{
  \vspace{-0.7cm}
  %\includegraphics[width=\textwidth]{TalkPics/invcomb021213/feyndiags}
  %% \begin{fmfgraph*}(100,70)
  %%         \fmfleft{i1,i2}
  %%         \fmfright{o1,o2,o3}
  %%         \fmf{fermion}{i1,v1,o1}
  %%         \fmf{fermion}{i2,v2,o3}
  %%         \fmf{phantom,tension=4/5}{v1,v2}
  %%         \fmffreeze
  %%         \fmf{photon,label=$W,,Z$}{v1,v3}
  %%         \fmf{photon,label=$W,,Z$}{v2,v3}
  %%         \fmf{dashes}{v3,o2}
  %%         \fmflabel{$q$}{i1}
  %%         \fmflabel{$q$}{i2}
  %%         \fmflabel{$q$}{o1}
  %%         \fmflabel{$q$}{o3}
  %%         \fmflabel{$H$}{o2}
  %%       \end{fmfgraph*}
}
\date{}
\begin{document}
\begin{fmffile}{higgsexoupdatefeyndiags}
\tikzstyle{every picture}+=[remember picture]

%TITLE PAGE
\section{Title}
\begin{frame}
  \titlepage
  
\end{frame}

\begin{frame}
  \frametitle{Overview}
  \begin{block}{}
    \begin{itemize}
    \item Update given last week on run II ntuples:
    \item Leptons and photons were in a good state
    \item Jets were in progress:
    \item[-] ak4PFCHS jets were in and verified
    \item[-] ak4PF jets need to wait until 74X
    \item type 1 MET was added, other METs will have to wait for JME recipes

    \end{itemize}
    \end{block}
\end{frame}

\begin{frame}
  \frametitle{Generator information recap}
  \begin{block}{}
    \begin{itemize}
    \item MiniAOD has two gen information collections:
    \item prunedGenCandidates: full info on a limited set of gen particles
    \item[-] Currently contains all leptons and b quarks
    \item[-] Evolving rapidly
    \item packedGenCandidates: packed info on all status 1
    \item[-] Mainly for clustering gen jets
    \item Currently working on storing the information we need
    \end{itemize}
  \end{block}
\end{frame}

\begin{frame}
  \frametitle{Generator information progress}
  \vspace{-.3cm}
  \begin{block}{}
    \begin{itemize}
    \item Gen jets and gen particles now stored
    \item Gen particles collection found to contain neutrinos and BSM particles
    \item[-] This is being fixed for 74X
    \item Reclustered gen jets very similar to stored ones
    \item[-] differences in the 4th SF
    \end{itemize}
  \end{block}
  \centering
    \includegraphics[width=.43\textwidth]{/home/patrick/genjets.pdf}
    \includegraphics[width=.43\textwidth]{/home/patrick/Genjet_pt_miniaod.pdf}

\end{frame}

\begin{frame}
  \frametitle{Trigger and tracks}
  \begin{block}{}
    \begin{itemize}
    \item We store trigger paths and HLT objects
    \item L1 extra information now also stored
    \item We were going to store track information for a variable Joao suggested
    \item[-] Proportion of tracks from PV included in our tag jets
    \item Variable can also be implemented by storing charged PF candidates from the PV
    \item[-] These are present in miniAOD by default and are easy to put into our ntuples
    \end{itemize}
  \end{block}
\end{frame}


\begin{frame}
  \frametitle{Summary}
  \label{lastframe}
  \begin{block}{}

    \begin{itemize}
    \item All objects have at least a basic recipe in our ntuples
    \item ak4 non-CHS jets and MET need most work
    \item[-] Chayanit has said she'll keep us up to date on MET progress
    \item First production of signal and QCD samples completed
    \item Updating scripts to work with crab 3 outputs and new ntuples
    \item[-] Once this is done we can try to make light trees and exercise the full chain
    \end{itemize}
  \end{block}
\end{frame}

\begin{frame}
  \frametitle{Backup}
\end{frame}

\begin{frame}
  \frametitle{Reminder of framework structure}

  \tikzstyle{format} = [draw, thin, fill=blue!20]
  \tikzstyle{medium} = [ellipse, draw, thin, fill=green!20, minimum height=2.5em]
  \begin{block}{}
    \begin{itemize}
    \item Focus today on miniAOD to ntuples
    \end{itemize}
  \end{block}

  \begin{tikzpicture}[auto,>=latex', thick]
    \path[->] node [format] (AOD) {AOD};
    \path[->] node [format,below of= AOD] (miniAOD) {miniAOD};
    \path[->] node [format,right= 2cm of {$(AOD)!0.5!(miniAOD)$}] (ntuple) {ICHiggsTauTau ntuples}
                  (AOD) edge node {grid} (ntuple);
    \path[->] [ultra thick,red](miniAOD) edge node {grid} (ntuple);
    \path[->] node [format,right= 2cm of ntuple] (lighttree) {Light trees};
    \path[->] (ntuple) edge node {batch job} (lighttree);
    \path[->] node [medium,below= 2cm of {$(ntuple)!0.5!(lighttree)$}] (output) {plots, yields, datacards etc.};
    \path[->] (ntuple) edge node {batch job} (output);
    \path[->] (lighttree) edge node {batch job} (output);
  \end{tikzpicture}
\end{frame}

\begin{frame}
  \frametitle{Electrons}
  \begin{block}{}
    \begin{itemize}
    \item In run 1 we used cut based identification at veto and tight working points
    \item Updated for run 2
    \item[-] Ntuples already contain all required variables
    \item[-] istight() etc. functions will be updated to new cut values
    \item[-] Details \href{https://twiki.cern.ch/twiki/bin/viewauth/CMS/CutBasedElectronIdentificationRun2}{here}
    \end{itemize}
  \end{block}
\end{frame}

\begin{frame}
\frametitle{Muons}
  \begin{block}{}
    \begin{itemize}
    \item In run 1 we used cut based identification at loose and tight working points
    \item[-] These are currently unchanged for run 2
    \item In run 2 there is also a medium working point with better fake rejection than loose but still high efficiency
    \item[-] Ntuples have been updated to contain variables required
    \item Updated for run 2, code has been updated to store all needed variables in ntuples
    \item[-] Details \href{https://twiki.cern.ch/twiki/bin/view/CMS/SWGuideMuonId2015}{here}
    \end{itemize}
  \end{block}
\end{frame}
  
\begin{frame}
  \frametitle{Taus}
  \begin{block}{}
    \begin{itemize}
    \item In run 1 we used same ID as $H\rightarrow\tau\tau$ group
    \item Baseline ID to be used by $H\rightarrow\tau\tau$ in run 2 currently being implemented
    \item[-] Details \href{https://twiki.cern.ch/twiki/bin/viewauth/CMS/CutBasedElectronIdentificationRun2}{here}
    \end{itemize}
  \end{block}
  
\end{frame}

\begin{frame}
  \frametitle{Jets}
  \begin{block}{}
    \begin{itemize}
    \item In run 1 we used ak5 non-CHS jets
    \item Switching to ak4 for run 2
    \item Only CHS jets are stored in miniAOD
    \item[-] We can remake non-CHS jets from packed candidates but no pu jet ID available until CMSSW\_7\_4\_X
    \item ak4PFCHS jets reclustered from packedCandidates now verified same as those in miniAOD
    \item[-] Gives confidence for remaking ak4PF jets without CHS
    \item B tag information is stored
    \end{itemize}
  \end{block}

\end{frame}

\begin{frame}
  \frametitle{MET}
  \begin{block}{}
    \begin{itemize}
    \item In run 1 we used type0PC+type1 corrected MET
    \item type 1 corrected MET is stored in miniAOD
    \item Can remake raw PF met from packed candidates
    \item[-] No recipe to go from this to type0+1, Chayanit investigating
    \item JetMET may recommend use of MVA met
    \item[-] TauTau group use this already so we should be able to implement it as well
    \end{itemize}
  \end{block}

\end{frame}

\begin{frame}
  \frametitle{Photons}
  \begin{block}{}
    \begin{itemize}
    \item Not used in run 1
    \item For run 2 we aim to use a $\gamma$+jets region
    \item Variables needed for POG cut based photon ID are now stored
    \item[-] Details \href{https://twiki.cern.ch/twiki/bin/view/CMS/CutBasedPhotonIdentificationRun2}{here}
    \end{itemize}
  \end{block}
\end{frame}

\end{fmffile}
\end{document}
