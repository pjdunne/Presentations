\documentclass[hyperref=colorlinks]{beamer}
\mode<presentation>
\usetheme{iclpt}
\setbeamertemplate{navigation symbols}{}
\setbeamertemplate{headline}{
  \begin{beamercolorbox}[leftskip=.2cm,rightskip=.2cm,topskip=.2cm,ht=1.1cm,dp=0.1cm,wd=\textwidth]{institute in head/foot}
    \includegraphics[height=1cm]{icl.pdf}
    \hfill
%    \includegraphics[height=1cm]{../Pics/ATLAS-Logo-Square-Blue-RGB.png}
    \includegraphics[height=1cm]{../Pics/CMS-Color.pdf}
  \end{beamercolorbox}
}
\setbeamertemplate{footline}{
  \begin{beamercolorbox}[ht=.35cm,dp=0.2cm,wd=\textwidth,leftskip=.3cm]{author in head/foot}%
    \begin{minipage}[c]{5cm}%
      \usebeamerfont{author in head/foot}
      \insertshortauthor 
      \insertshorttitle
    \end{minipage}\hfill%
    \hfill
    \insertframenumber{} / \ref{lastframe}
    %\hfill
    \begin{minipage}{6cm}
      \hfill
      %\insertshorttitle
    \end{minipage}
  \end{beamercolorbox}%
}

\definecolor{beamer@icdarkblue}{RGB}{0,51,102}
\definecolor{beamer@icmiddleblue}{RGB}{0,82,150} 
\definecolor{beamer@iclightblue}{RGB}{200,212,232}
\definecolor{beamer@icmiddlered}{RGB}{204,51,0}
\definecolor{beamer@iclightred}{RGB}{232,212,32}

\usepackage{tikz}
\usetikzlibrary{arrows,shapes,backgrounds}
\usepackage{color}
\usepackage{tabularx,colortbl}
\usepackage{graphicx}
\usepackage{pdfpages}
\usepackage{feynmp}
\usepackage{rotating}
\usepackage{moresize}
\usepackage{slashed}
\usepackage{xcolor,colortbl}
\DeclareGraphicsRule{*}{mps}{*}{}
\hypersetup{colorlinks=false}

\title[Searches for invisibly decaying Higgs bosons]{\vspace{-0.2cm} Searches for invisibly decaying Higgs bosons}
\author[P. Dunne]{Patrick Dunne - Imperial College London \\ RHUL 05/10/2016}
\titlegraphic{
  \vspace{-0.4cm}
  \begin{fmffile}{dmlhcfeyndiagstitle}
    \begin{fmfgraph*}(75,75)
      \fmfleft{i0,i2,ix,i3,i5}
      \fmfright{o0,o3,o1,o4,o6}
      \fmf{phantom,tension=4/3}{i2,v1,o3}
      \fmf{phantom,tension=4/3}{i3,v2,o4}
      \fmffreeze
      \fmf{gluon,tension=4/3}{i2,v1}
      \fmf{gluon,tension=4/3}{i3,v2}
      \fmf{fermion,tension=0}{v1,v2}
      \fmf{fermion,tension=2/3}{v2,v3,v1}
      \fmf{dashes}{v3,o1}
      \fmflabel{$g$}{i2}
      \fmflabel{$g$}{i3}
      \fmflabel{$H$}{o1}
    \end{fmfgraph*}    
    \hspace{.5cm}
    \begin{fmfgraph*}(75,75)
      \fmfleft{i1,i2}
      \fmfright{o1,o2,o3}
      \fmf{fermion}{i1,v1,o1}
      \fmf{fermion}{i2,v2,o3}
      \fmf{photon,label=$W,,Z$}{v1,v3}
      \fmf{photon,label=$W,,Z$}{v2,v3}
      \fmf{dashes}{v3,o2}
      \fmflabel{$q$}{i1}
      \fmflabel{$q$}{i2}
      \fmflabel{$q$}{o1}
      \fmflabel{$q$}{o3}
      \fmflabel{$H$}{o2}
    \end{fmfgraph*}
    \hspace{.5cm}
    \begin{fmfgraph*}(75,75)
      \fmfleft{i1,i2}
      \fmfright{o1,o2}
      \fmf{fermion}{i1,v1}
      \fmf{fermion}{v1,i2}
      \fmf{photon,label=$W,,Z$}{v1,v2}
      \fmf{photon}{v2,o1}
      \fmf{dashes}{v2,o2}
      \fmflabel{$q$}{i1}
      \fmflabel{$\bar{q}$}{i2}
      \fmflabel{$W,Z$}{o1}
      \fmflabel{$H$}{o2}
    \end{fmfgraph*}
  \end{fmffile}
  %% \begin{fmfgraph*}(100,70)
  %%         \fmfleft{i1,i2}
  %%         \fmfright{o1,o2,o3}
  %%         \fmf{fermion}{i1,v1,o1}
  %%         \fmf{fermion}{i2,v2,o3}
  %%         \fmf{phantom,tension=4/5}{v1,v2}
  %%         \fmffreeze
  %%         \fmf{photon,label=$W,,Z$}{v1,v3}
  %%         \fmf{photon,label=$W,,Z$}{v2,v3}
  %%         \fmf{dashes}{v3,o2}
  %%         \fmflabel{$q$}{i1}
  %%         \fmflabel{$q$}{i2}
  %%         \fmflabel{$q$}{o1}
  %%         \fmflabel{$q$}{o3}
  %%         \fmflabel{$H$}{o2}

  %%       \end{fmfgraph*}
}
\date{}
\begin{document}
\tikzstyle{every picture}+=[remember picture]
\tikzstyle{na} = [baseline=-.5ex]
\begin{fmffile}{dmlhcfeyndiags}


  %TITLE PAGE
  %20 mins + 5 questions
  \section{Title}
  \begin{frame}
    \titlepage
  \end{frame}

  \begin{frame}
    \frametitle{Outline}
    \begin{block}{}
      \begin{itemize}
      \item Why search for invisibly decaying Higgs bosons
      \item How to search for invisibly decaying Higgs bosons:
      \item[-] Direct and indirect searches
      \item[-] Focus on most sensitive VBF channel
      \item Results from LHC Runs 1 and 2
      \item Projections of future sensitivity
      \end{itemize}
    \end{block}
  \end{frame}

  %??new unconstrained thing, what we know so far, lots of room left  
  \begin{frame}
    \frametitle{Why look for invisibly decaying Higgs bosons?}
    \begin{columns}
    \column{.55\textwidth}
    \vspace{-.2cm}    
    \begin{block}{New particle}
      \small
      \begin{itemize}
      \item Measurements of the Higgs boson made so far are impressive:
        \vspace{-.1cm}
      \item[-] Mass measured with 0.2\% error
      \item But a lot of parameters are still relatively unconstrained:
        \vspace{-.1cm}
      \item[-] Indirect limit on width is $\sim$4$\Gamma_{SM}$
      \item Plenty of room for Higgs boson couplings to exotic particles
      \end{itemize}
    \end{block}
    \column{.45\textwidth}
    \includegraphics[width=.95\textwidth]{TalkPics/DM@LHC2016/CMS-PAS-HIG-15-002_Figure_012.pdf}
      \centering
      \scriptsize

      CMS-PAS-HIG-15-002
      
      ATLAS-CONF-2015-044
    \end{columns}
  \end{frame}

  \begin{frame}
    \frametitle{Why invisible particularly?}
    \begin{block}{}
      \begin{itemize}
      \item There is evidence from several sources for ``dark matter'' (DM)
      \item Amounts of DM and SM matter seem to be of the same order of magnitude so it's likely they interact
      \item Where does dark matter get its mass from?
      \end{itemize}
    \end{block}
    \includegraphics[clip=true,trim=0 0 0 0,height=.5\textheight,width=.5\textwidth]{TalkPics/sgs120315/bulletcluster.png}
    \includegraphics[clip=true,trim=0 0 0 0,height=.5\textheight,width=.5\textwidth]{TalkPics/sgs120315/rotationcurve.jpg}
  \end{frame}
  

  %??Expand with mass term lagrangian and some DM intro
  \begin{frame}
    \frametitle{Why look for invisibly decaying Higgs bosons?}
    \vspace{-.3cm}
    \begin{columns}
      \column{1.06\textwidth}
    \begin{block}{Theoretical Motivations}
      \small
      \begin{itemize}
      \item All massive SM particles get their mass through Higgs boson couplings:
      \item[-] Fermions: $ \bar{\psi}_{i}y_{ij}\psi_{j}\phi+$ hermitian conjugate
      \item[-] Bosons: $|D_{\mu}\phi|^2-V(\phi)$
      \item DM must also get its mass from somewhere motivating interaction with a scalar like the Higgs
      \item Some theories of collider DM also have similar production mechanisms to the Higgs
      \end{itemize}
    \end{block}
    %??feynman diagram of VBF
    \end{columns}
  \end{frame}
  

  %??expand a bit more, decisions on fixed width etc.
  \begin{frame}
    \frametitle{How to search for invisibly decaying Higgs bosons}
    \vspace{-.2cm}
    \begin{columns}
      \column{.5\textwidth}
      \begin{block}{Indirect searches}
          \small
          \begin{itemize}
          \item Compare visible width to total width:
          \item[-] $\rm{BR}_{BSM}=\frac{\Gamma_{H}-\Gamma_{vis}}{\Gamma_{H}}$
          \item No measurement of $\Gamma_{H}$, need to make an assumption
          \item[-] Usually assume SM width
          \item ATLAS+CMS combination gives an observed (expected) limit on $\rm{BR}_{BSM}$ of 0.34 (0.35)
          \end{itemize}
      \end{block}
      \column{.5\textwidth}
      \includegraphics[width=\textwidth]{TalkPics/DM@LHC2016/CMS-PAS-HIG-15-002_Figure_015.pdf}
      \centering
      \scriptsize

      CMS-PAS-HIG-15-002
      
      ATLAS-CONF-2015-044
       \end{columns}
       %ATLAS or CMS rates of each channel plot
  \end{frame}

  %??one slide on each process
  \begin{frame}
    \frametitle{How to search for invisibly decaying Higgs bosons}
    \begin{columns}
      \column{1.06\textwidth}
    \begin{block}{Direct searches}
      \small
      \begin{itemize}
      \item Look for products of associated production of a Higgs boson plus momentum imbalance ``$E_{T}^{miss}$''
      \end{itemize}
    \end{block}
    \end{columns}
    \begin{columns}
      \column{.5\textwidth}
      %??diagram of MET
      \column{.5\textwidth}
      \includegraphics[width=\textwidth]{TalkPics/DM@LHC2016/XS_8TeV-eps-converted-to.pdf}
      \end{columns}
    \end{frame}

  %??ggh
  \begin{frame}
    \frametitle{Gluon Fusion: ggH}
      \begin{block}{Production channels}
          \small
          %??high rate low purity
          %??ggH, VBF, VH tikz lines to plot if time
          \begin{itemize}
          \item VBF mode is most sensitive
          \item[-] Second highest rate and distinctive topology
          \item Gluon fusion has no visible products, needs ISR
          \item[-] High rate, difficult final state
          \item VH has clean final states but low rate
          \end{itemize}
      \end{block}

  \end{frame}

  %??zh
  \begin{frame}
    \frametitle{Vector boson associated production: VH}
    %??high purity low rate
  \end{frame}

  %??vbf
  \begin{frame}
    \frametitle{Vector Boson Fusion: VBF}
    %??best of both worlds
  \end{frame}

  %??ttH
  \begin{frame}
    \frametitle{ttH}
    %??aside just for completeness
  \end{frame}


  %??why ATLAS and CMS are good for invisible detection
  \begin{frame}
    \frametitle{Detector capabilities}
    %??things needed to see VBF i.e. jets and met
  \end{frame}

  %??bit on particle flow
  \begin{frame}
  \end{frame}

  %??bit on met and jet corrections
  \begin{frame}
  \end{frame}

  %??bit on met and jet corrections
  \begin{frame}
  \end{frame}

  %??What are the main backgrounds to VBF: QCD
  \begin{frame}
  \end{frame}

  %??What are the main backgrounds to VBF: V+jets
  \begin{frame}
  \end{frame}

  %??What are the main backgrounds to VBF: others
  \begin{frame}
  \end{frame}

  %??Outline challenges of VBF analysis trigger/high QCD background
  \begin{frame}
  \end{frame}

  %??parked data
  \begin{frame}
  \end{frame}

  %??Strategy after trigger
  \begin{frame}
  \end{frame}


  %??Selection: preselection variables
  \begin{frame}
  \end{frame}

  %??Selection: selection optimisation
  \begin{frame}
  \end{frame}

  %??Selection: why not a BDT
  \begin{frame}
  \end{frame}

  %??Remaining background estimation general data driven methods intro
  \begin{frame}
  \end{frame}

  %??Remaining background estimation W/Z
  \begin{frame}
  \end{frame}

  %??Remaining background estimation QCD, how it evolved
  \begin{frame}
  \end{frame}

  %??Remaining background estimation QCD, how it evolved
  \begin{frame}
  \end{frame}

  %??Results over run 1 and statistics: prompt
  \begin{frame}
  \end{frame}

  \begin{frame}
    %??HIG-14-038
    \frametitle{Run 1 CMS direct searches - VBF}%??parked
    \begin{columns}
      %??quick description of parked data  
      \column{.5\textwidth}
      \vspace{-.3cm}
      \begin{block}{}
        \small
        \vspace{-.2cm}
        \begin{itemize}
        \item Select two jets with large $\Delta\eta$ separated from large $E_{T}^{miss}$
          \vspace{-.2cm}
        \item Update of first search Eur. Phys. J. C 74 (2014) 2980
          \vspace{-.2cm}
        \item Dedicated ``parked data'' trigger
          \vspace{-.2cm}
        \item Counting experiment with data driven backgrounds
          \vspace{-.2cm}
        \item V+jets backgrounds separately normalised
          \vspace{-.2cm}
        \item Observed (expected) limit on $\mathcal{B}\left(H\rightarrow inv.\right)$ for $m_{H}=$125 GeV is 57 (40)\%
          \vspace{-.2cm}
        \item[-] If all normalisations had same uncertainty as $W\rightarrow\mu\nu$ expected limit would be 33\%
        \end{itemize}
      \end{block}
      \column{.5\textwidth}
      %??limit plot from hig-14-038
      \includegraphics[width=\textwidth]{TalkPics/DM@LHC2016/Figure_007-a.pdf}
      \centering
      \scriptsize
      
      CMS-PAS-HIG-14-038
    \end{columns}
  \end{frame}


  %??intro to ZH searches

  %??CMS Run 1
  \begin{frame}
    %HIG-13-030
    \frametitle{Run 1 CMS direct searches - ZH}
    \begin{columns}
      %quick description comb from hig-13-030
      \column{.5\textwidth}
      \begin{block}{}
        \small
        \begin{itemize}
        \item Searches in $Z\rightarrow\ell\ell$ and $Z\rightarrow b\bar{b}$ channels
        \item $Z(\ell\ell)H$ search is a 2D shape analysis with data driven backgrounds
        \item $Z(b\bar{b})H$ search is a BDT shape analysis with data driven backgrounds
        \item Combined $ZH$ searches observed (expected) limit on $\mathcal{B}\left(H\rightarrow inv.\right)$ for $m_{H}=$125 GeV is 81 (83)\%
        \end{itemize}
      \end{block}
      \column{.5\textwidth}
      \includegraphics[width=\textwidth]{TalkPics/DM@LHC2016/Fig9b-ZH-LimitNorm.pdf}      
      \centering
      \scriptsize
      
      Eur. Phys. J. C 74 (2014) 2980
    \end{columns}
  \end{frame}

  \begin{frame}
    %??EXO-12-055
    \frametitle{Run 1 CMS direct searches - Monojet+V(had)H}
    \begin{columns}
      %??quick description
      \column{.5\textwidth}
      \begin{block}{}
        \small
        \begin{itemize}
        \item Search has categories targeting $V(had)H$ and $ggH$ production modes
        \item $E_{T}^{miss}$ shape analysis with data driven background estimation
        \item Observed (expected) limit on $\mathcal{B}\left(H\rightarrow inv.\right)$ for $m_{H}=$125 GeV is 53 (62)\%
        \end{itemize}
      \end{block}
      \column{.5\textwidth}
      %??fig 8b from pas
      \includegraphics[width=\textwidth]{TalkPics/DM@LHC2016/CMS-PAS-EXO-12-055_Figure_008-b.png}
      \centering
      \scriptsize
      
      CMS-PAS-EXO-12-055
    \end{columns}
  \end{frame}

  %??why combine and how does CMS handle correlated errors
  \begin{frame}
  \end{frame}

  %??Combinations expand
  \begin{frame}
    %??HIG-15-012
    \frametitle{Run 1 CMS direct searches - Combination}
    \vspace{-.2cm}
    \begin{block}{}
      \small
      \begin{itemize}
        \vspace{-.1cm}
      \item Combine by production mode as well as full combination
        \vspace{-.2cm}
      \item[-] ggH-tagged is monojet, VH-tagged is Z($\ell\ell$)H+Z($bb$)H+V(had)H, VBF-tagged is VBF
      \item Obs. (exp.) limit on $\mathcal{B}\left(H\rightarrow inv.\right)$ at $m_{H}=$125 GeV is 36 (30)\%
      \end{itemize}
    \end{block}
    \begin{columns}
      %??by category plot
      \column{.5\textwidth}
      \includegraphics[width=.8\textwidth]{TalkPics/DM@LHC2016/CMS-PAS-HIG-15-012_Figure_002.png}
      \column{.5\textwidth}
      \includegraphics[width=.9\textwidth]{TalkPics/DM@LHC2016/CMS-PAS-HIG-15-012_Figure_003.png}
    \end{columns}
    \centering
    \scriptsize
    
    CMS-PAS-HIG-15-012
  \end{frame}


  %??Moving to a simultaneous fit for run 2
  \begin{frame}
  \end{frame}

  %??Current run 2 result and combinations
  \begin{frame}
    %??HIG-16-009
    \frametitle{Run 2 CMS direct searches - VBF}
    %??mass scan plot, WZ tying together
%    \begin{columns}
%      \column{.5\textwidth}
      \begin{block}{}
        \small
        \begin{itemize}
        \item Dedicated trigger used again
          \vspace{-.2cm}
        \item Counting experiment with data driven background estimation
          \vspace{-.2cm}
        \item V+jets backgrounds all taken to have same normalisation
          \vspace{-.2cm}
        \item Observed (expected) limit on $\mathcal{B}\left(H\rightarrow inv.\right)$ for $m_{H}=$125 GeV is 69 (62)\% 
        \end{itemize}
      \end{block}
      %\column{.5\textwidth}
      %??mass scan
      \centering

      \includegraphics[width=.5\textwidth]{TalkPics/DM@LHC2016/output_run2ana_160329_sig/nunu_alljetsmetnomu_mindphi.pdf}
      \includegraphics[width=.35\textwidth]{TalkPics/DM@LHC2016/brlhscan.pdf}
      \centering
      \scriptsize
      
      CMS-PAS-HIG-16-009
%    \end{columns}
  \end{frame}

  %??CMS Run 2
  \begin{frame}
    %??HIG-16-008
    \frametitle{Run 2 CMS direct searches - ZH}
    %??fig 3c, mention move away from mass scans in BR
    %\begin{columns}
      %\column{.5\textwidth}
      \begin{block}{}
        \small
        \begin{itemize}
        \item Targets $Z\rightarrow\ell\ell$ final state
          \vspace{-.2cm}
        \item 2D shape analysis with leading backgrounds estimated using MC
          \vspace{-.2cm}
        \item Observed (expected) limit on $\mathcal{B}\left(H\rightarrow inv.\right)$ for $m_{H}=$125 GeV is 124 (124)\%
        \end{itemize}
      \end{block}
      %\column{.5\textwidth}
      \includegraphics[width=.4\textwidth]{TalkPics/DM@LHC2016/HIG16008datamc.png}
      \includegraphics[width=.4\textwidth]{TalkPics/DM@LHC2016/CMS-PAS-HIG-16-008_Figure_003-c.png}
       \centering
      \scriptsize
      
      CMS-PAS-HIG-16-008
%    \end{columns}
  \end{frame}

  
  %??HIG-16-009 Combination CMS Run 1 + Run 2
  \begin{frame}
    \frametitle{Run 2 CMS direct searches - Combination}
    %??explain contributing analyses
    %??13 TeV only and 8 TeV+13 TeV category plot
    \begin{block}{}
      \begin{itemize}
      \item First CMS analysis combining 8 and 13 TeV results
      \item Limit calculated both by production mode and overall
      \item Combined observed (expected) limit on $\mathcal{B}\left(H\rightarrow inv.\right)$ for $m_{H}=$125 GeV is 32 (26)\% %??
      \end{itemize}
      
    \end{block}
    \begin{columns}
      \column{.5\textwidth}
      \includegraphics[width=\textwidth]{TalkPics/DM@LHC2016/channellimit13_withoutMono.pdf}
      \column{.5\textwidth}
      \includegraphics[width=\textwidth]{TalkPics/DM@LHC2016/channellimit.pdf}
    \end{columns}
    \centering
    \scriptsize
    
    CMS-PAS-HIG-16-009
  \end{frame}

  

  %??DM pheno paper
  \begin{frame}
    \frametitle{Projections}
    \begin{columns}
      \column{.5\textwidth}
      \begin{block}{}
        \small
        \begin{itemize}
        \item CMS VBF analysis projected to increased luminosity at 13 TeV
        \item If systematics scale as $\sqrt{\mathcal{L}}$ can exclude $\mathcal{B}\left(H\rightarrow inv.\right)=5\%$ with full LHC dataset
        \end{itemize}
      \end{block}
      \column{.5\textwidth}
      \includegraphics[width=\textwidth]{TalkPics/DM@LHC2016/phenoprojectedvbflimit.pdf}
      \centering
      \scriptsize

      arXiv:1603.07739
    \end{columns}
  \end{frame}

  \begin{frame}
    \frametitle{Dark matter interpretations - Run 1 results}
    %??ATLAS DM plot and arxiv EFT plane plot
    \begin{columns}
      \column{.5\textwidth}
      \begin{block}{}
        \small
        \begin{itemize}
        \item Several models for interpreting invisible Higgs limits
          \vspace{-.1cm}
        \item ``Higgs portal'' has been quite commonly used
          \vspace{-.2cm}
        \item[-] Assume 125 GeV Higgs acts as mediator between visible and dark matter sectors
          \vspace{-.2cm}
        \item[-] Assume scalar, fermion or vector dark matter
          \vspace{-.1cm}
        \item Scalar dark matter would mix with Higgs boson
          \vspace{-.2cm}
        \item[-] mixing angle must be small
          \vspace{-.1cm}
        \item Vector dark matter width goes to infinity as mass decreases
        \end{itemize}
      \end{block}      

      \column{.5\textwidth}

      \includegraphics[width=\textwidth]{TalkPics/DM@LHC2016/ATLASdmlimit.png}
      \centering
      \scriptsize

      JHEP11(2015)206      
    
    \end{columns}
  \end{frame}

  \begin{frame}
    \frametitle{Dark matter interpretations - Projections}
    \begin{block}{}
      \small
      \begin{itemize}
      \item Models with electroweak couplings studied: favour VBF channel
      \item VBF topology allows looser $E_{T}^{miss}$ selection making EFTs valid
      \item Also investigate simplified models with scalar/pseudoscalar mediator
      \item Projections of CMS VBF channel sensitivity at several luminosities
      \end{itemize}
    \end{block}
    \begin{columns}
      \column{.5\textwidth}
      \includegraphics[width=\textwidth]{TalkPics/DM@LHC2016/D5_multilumi.pdf}
      \column{.5\textwidth}
      \includegraphics[width=\textwidth]{TalkPics/DM@LHC2016/Aplane.pdf}      
    \end{columns}
      \centering
      \scriptsize

      arXiv:1603.07739
  \end{frame}

  %??three more slides on projections
  \begin{frame}
  \end{frame}
  
  \begin{frame}
  \end{frame}

  \begin{frame}
  \end{frame}

  \begin{frame}
    \frametitle{Summary}
    \label{lastframe}
    \begin{block}{}
      \begin{itemize}
      \item Both collaborations are sensitive to $\mathcal{B}\left(H\rightarrow inv.\right)\sim 25\%$ with current datasets
      \item[-] Current 95\% CL upper observed (expected) limits from direct searches are CMS: 32 (26)\%, ATLAS: 25 (27) \%
      \item[-] Combination of channels allow sensitivity to be greatly improved
      \item Projected limit on $\mathcal{B}\left(H\rightarrow inv.\right)$ $\sim$10-20\% from VBF alone by the end of LHC Run 2 and 5\% by end of LHC running assuming systematics scale as $\sqrt{\mathcal{L}}$
      \end{itemize}
    \end{block}
  \end{frame}


  %??ATLAS RUN 1 
  \begin{frame}
    %ZH http://journals.aps.org/prl/abstract/10.1103/PhysRevLett.112.201802
    \frametitle{Run 1 ATLAS direct searches - Z($\ell\ell$)H}
    %search description and limit
    \begin{columns}
      \column{.5\textwidth}
      \begin{block}{}
        \small
        \begin{itemize}
        \item Selects two leptons opposite large $E_{T}^{miss}$
        \item $E_{T}^{miss}$ shape analysis with data driven backgrounds
        \item Observed (expected) limit on $\mathcal{B}\left(H\rightarrow inv.\right)$ for $m_{H}=$125.5 GeV is 75 (62)\%
        \end{itemize}
      \end{block}
      \column{.5\textwidth}
      \includegraphics[width=\textwidth]{TalkPics/DM@LHC2016/ATLASZH.png}
      \centering
      \scriptsize
      
      PRL 112, 201802 (2014)
    \end{columns}
  \end{frame}

  \begin{frame}
    %V(had)H: http://arxiv.org/abs/1504.04324
    \frametitle{Run 1 ATLAS direct searches - V(had)H}
    %search description and limit
    \begin{columns}
      \column{.5\textwidth}
      \begin{block}{}
        \small
        \begin{itemize}
        \item Targets $W/Z\rightarrow qq$ final state
        \item $E_{T}^{miss}$ and dijet $p_{T}$ shape analysis with data driven backgrounds
        \item Observed (expected) limit on $\mathcal{B}\left(H\rightarrow inv.\right)$ for $m_{H}=$125 GeV is 78 (86)\%
        \end{itemize}
      \end{block}
      \column{.5\textwidth}
      \includegraphics[width=\textwidth]{TalkPics/DM@LHC2016/ATLASVH.png}
      \centering
      \scriptsize

      Eur. Phys. J. C (2015) 75:337
    \end{columns}
  \end{frame}

  \begin{frame}
    %http://arxiv.org/pdf/1508.07869v2.pdf
    \frametitle{Run 1 ATLAS direct searches - VBF}
    %search description and limit mention tying W and Z together
    \begin{columns}
      \column{.5\textwidth}
      \begin{block}{}
        \small
        \begin{itemize}
        \item Select two jets with large $\Delta\eta$ opposite large $E_{T}^{miss}$
        \item Counting experiment with data driven background estimation
        \item[- ] $W\rightarrow e\nu$, $W\rightarrow \mu\nu$ and $W\rightarrow \tau\nu$ and $Z\rightarrow \nu\nu$ normalisation tied together
        \item Observed (expected) limit on $\mathcal{B}\left(H\rightarrow inv.\right)$ for $m_{H}=$125 GeV is 28 (31)\%
        \end{itemize}
      \end{block}
      \column{.5\textwidth}
      %??no mas scan find another pic to use
      \includegraphics[width=\textwidth]{TalkPics/DM@LHC2016/ATLASvbfyields.png}
      \centering
      \scriptsize

      JHEP 01 (2016) 172
    \end{columns}
  \end{frame}

  \begin{frame}
    %http://arxiv.org/pdf/1509.00672v2.pdf (fig 8)
    %comb with vis
    \frametitle{Run 1 ATLAS direct searches - Combination}
    \begin{columns}
      \column{.5\textwidth}
      \begin{block}{}
        \small
        \begin{itemize}
        \item Combining searches significantly improves limits
        \item Direct searches provide most sensitivity
        \item[-] Observed (expected) limit on $\mathcal{B}\left(H\rightarrow inv.\right)$ for $m_{H}=$125 GeV is 25 (27)\%
        \item Adding indirect results adds assumption on Higgs total width 
        \item[-] Observed (expected) limit on $\mathcal{B}\left(H\rightarrow inv.\right)$ for $m_{H}=$125 GeV is 23 (24)\%
        \end{itemize}
      \end{block}
      \column{.5\textwidth}
      \includegraphics[width=\textwidth]{TalkPics/DM@LHC2016/ATLASviscomb.png}
      \centering
      \scriptsize

      JHEP11(2015)206
    \end{columns}
  \end{frame}





  
\end{fmffile}
\end{document}

