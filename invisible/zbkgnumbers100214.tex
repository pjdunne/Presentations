\documentclass[hyperref=colorlinks]{beamer}
\mode<presentation>
\usetheme{iclpt}
\setbeamertemplate{navigation symbols}{}
\setbeamertemplate{headline}{
\begin{beamercolorbox}[leftskip=.2cm,rightskip=.2cm,topskip=.2cm,ht=1.1cm,dp=0.1cm,wd=\textwidth]{institute in head/foot}
  \includegraphics[height=1cm]{icl.pdf}
  \hfill
  \includegraphics[height=1cm]{../Pics/CMS-Color.pdf}
\end{beamercolorbox}
}
\setbeamertemplate{footline}{
\begin{beamercolorbox}[ht=.55cm,dp=0.4cm,wd=\textwidth,leftskip=.3cm]{author in head/foot}%
  \begin{minipage}[c]{5cm}%
    \usebeamerfont{author in head/foot}
    \insertshortauthor 
    \insertshorttitle
    \end{minipage}\hfill%
  \insertframenumber{} / \pageref{lastframe}
  \hfill
  \begin{minipage}{6cm}
    \hfill
  \end{minipage}
\end{beamercolorbox}%
}

\usepackage{color}
\usepackage{tabularx,colortbl}
\usepackage{graphicx}
\usepackage{pdfpages}
\usepackage{feynmp}
\DeclareGraphicsRule{*}{mps}{*}{}

\title{\vspace{-0.2cm} Z Background Formulae}
\subtitle{Paper - HIG-13-030, PASs: HIG-13-013, HIG-13-018, HIG-13-028 \vspace{-0.7cm}}
\author[P. Dunne]{D. Colling, G. Davies \underline{P. Dunne}, A.M. Magnan, A. Nikitenko and Joao Pela: Imperial with \\ R. Aggleton, J. Brooke: Bristol \\ C.Asawangtrakuldee, Q.Li: Peking \\ P. Srimanobhas: Chulalongkorn \\ S. Kumar, K. Mazumdar: Mumbai}
\titlegraphic{
  \vspace{-0.7cm}
  %% \begin{fmfgraph*}(100,70)
%%         \fmfleft{i1,i2}
%%         \fmfright{o1,o2,o3}
%%         \fmf{fermion}{i1,v1,o1}
%%         \fmf{fermion}{i2,v2,o3}
%%         \fmf{phantom,tension=4/5}{v1,v2}
%%         \fmffreeze
%%         \fmf{photon,label=$W,,Z$}{v1,v3}
%%         \fmf{photon,label=$W,,Z$}{v2,v3}
%%         \fmf{dashes}{v3,o2}
%%         \fmflabel{$q$}{i1}
%%         \fmflabel{$q$}{i2}
%%         \fmflabel{$q$}{o1}
%%         \fmflabel{$q$}{o3}
%%         \fmflabel{$H$}{o2}
%%       \end{fmfgraph*}
}
\date{}
\begin{document}
\begin{fmffile}{feynmandiagszbkg}

%TITLE PAGE
\section{Title}
\begin{frame}
  \titlepage
  
\end{frame}

%OUTLINE
\begin{frame}
  \frametitle{Definitions}
  \vspace{-.1cm}

  \begin{block}{\scriptsize Regions}
    \begin{itemize}
      \scriptsize
    \item Control Region: Reco dimuon, with $60<m_{\mu\mu}^{reco}<120 GeV$ passing VBF selections
    \item[-] $N_{C}^{MC}$ is measured in $Z\rightarrow\mu\mu + Jets$ MC with a generator level cut of $m_{Z}^{Gen}>50 GeV$
    \item Signal Region: VBF selections and no veto leptons
    \item[-] We use the same $Z\rightarrow\mu\mu$ sample as for $N_{C}^{MC}$ and ignore the leptons to approximate a $Z\rightarrow\nu\nu$ sample, this will be denoted $N_{S}^{MC}$
    \item[-] For the efficiencies to be the same for the $Z\rightarrow\mu\mu$ and $Z\rightarrow\nu\nu$ samples a generator level mass window of $60<m_{Z}^{Gen}<120$ must be applied, this will be denoted $N_{S}^{MC}[60,120]$.
    \end{itemize}
  \end{block}
\end{frame}

\begin{frame}
  \frametitle{Derivation of formula}
  \vspace{-.2cm}
  \begin{block}{}
    \begin{itemize}
    \item {\scriptsize Basic formula for data driven estimate}
    \end{itemize}
    \footnotesize
    \hspace{1cm}$N_{S}^{\nu\nu\,Data}=\frac{N_{C}^{Data}-N_{C}^{BKG}}{N_{C}^{MC}}\cdot N_{S}^{\nu\nu\,MC}$
    \begin{itemize}
    \item {\scriptsize To use $Z\rightarrow\mu\mu$ MC we use the formula:}
    \end{itemize}
    
        \hspace{1cm}{\scriptsize $N_{S}^{\nu\nu\,MC}=N_{S}^{MC}[60,120]\cdot\underbrace{\frac{\sigma(Z\rightarrow\nu\nu)}{\sigma(Z/\gamma^{*}\rightarrow\mu\mu,\,60<m_{Z}^{Gen}<120 GeV)}}_{R[60,120]}$}
        \begin{itemize}
    \item {\scriptsize The cross-section ratio that we have calculated is:}
        \end{itemize}
        
        \hspace{1cm}$R[50,\infty]=\frac{\sigma(Z\rightarrow\nu\nu)}{\sigma(Z/\gamma^{*}\rightarrow\mu\mu,\,m_{Z}^{Gen}>50)}$
        \begin{itemize}
    \item {\scriptsize We therefore use:}
      \end{itemize}
      \scriptsize
    \hspace{1cm}$\begin{aligned}[t] R[60,120]&=\frac{\sigma(Z/\gamma^{*}\rightarrow\mu\mu,m_{Z}^{Gen}>50 GeV)}{\sigma(Z/\gamma^{*}\rightarrow\mu\mu,\,60<m_{Z}^{Gen}<120 GeV)}\cdot R[50,\infty] \\ &=\frac{N(Z/\gamma^{*}\rightarrow\mu\mu,m_{Z}^{Gen}>50 GeV)}{N(Z/\gamma^{*}\rightarrow\mu\mu,60<m_{Z}^{Gen}<120 GeV)} \cdot R[50,\infty] \end{aligned}$
  \end{block}
\end{frame}

\begin{frame}
  \frametitle{Derivation of formula (2)}
  \begin{columns}
    \column{1.1\textwidth}
  \begin{block}{}
    \begin{itemize}
    \item {\scriptsize Substituting our expression for $N_{S}^{\nu\nu\,MC}$ into the original formula gives:}
    \end{itemize}
    \scriptsize
    $\begin{aligned}[t] N_{S}^{\nu\nu\,Data}&=\frac{N_{C}^{Data}-N_{C}^{BKG}}{N_{C}^{MC}}\cdot N_{S}^{MC}[60,120]\cdot R[60,120] \\
    &=\frac{N_{C}^{Data}-N_{C}^{BKG}}{N_{C}^{MC}}\cdot N_{S}^{MC}[60,120]\cdot \frac{N(Z/\gamma^{*}\rightarrow\mu\mu,m_{Z}^{Gen}>50 GeV)}{N(Z/\gamma^{*}\rightarrow\mu\mu,60<m_{Z}^{Gen}<120 GeV)} \cdot R[50,\infty]\end{aligned}$
  \end{block}
  \end{columns}
\end{frame}

\begin{frame}
\frametitle{Formulae from paper}
  \begin{block}{}
    \centering
        $N_{S}^{Data}=(N_{C}^{Data}-N_{C}^{BKG})\cdot R[50,\infty]\cdot\frac{\epsilon_{S}^{VBF}}{\epsilon_{C}^{VBF}\epsilon_{\mu\mu}}$
    \begin{itemize}
    \item $\epsilon_{\mu\mu} = \frac{N(\rm{Z/\gamma^{*}\rightarrow\mu\mu,\,reco\,dimuon},\,60<m_{\mu\mu}^{reco}<120 GeV)}{N(Z/\gamma^{*}\rightarrow\mu\mu,\,m_{Z}^{Gen}>50 GeV)}$
    \item $\epsilon_{C}^{VBF} = \frac{N_{C}^{MC}}{N(Z/\gamma^{*}\rightarrow\mu\mu,\,\rm{reco\,dimuon},\,60<m_{\mu\mu}^{reco}<120 GeV)}$
    \item $\epsilon_{S}^{VBF} = \frac{N_{S}^{MC}[60,120]}{N(Z/\gamma^{*}\rightarrow\mu\mu,\,60<m_{Z}^{Gen}<120 GeV)}$
    \item {\scriptsize n.b. efficiencies are not defined in the paper, so the differences in the denominator between $\epsilon_{\mu\mu}$ and $\epsilon_{S}^{VBF}$ are not apparent}
    \end{itemize}
  \end{block}
\end{frame}

\begin{frame}
  \frametitle{Simplifications}
  \begin{columns}
    \column{1.1\textwidth}
    \vspace{-0.2cm}
  \begin{block}{}
  \begin{itemize}
  \item {\scriptsize Numerator of $\epsilon_{\mu\mu}$ and denominator of $\epsilon_{C}^{VBF}$ cancel so they should not be included in the error calculation}
  \item[-] {\scriptsize Currently stat, lepton ID, JES, JER and UES uncertainties are considered on all terms}
  \item $\epsilon_{C}^{VBF}\cdot\epsilon_{\mu\mu} = \frac{N_{C}^{MC}}{N(Z/\gamma^{*}\rightarrow\mu\mu,\,m_{Z}^{Gen}>50 GeV)}$
  \item $\frac{\epsilon_{S}^{VBF}}{\epsilon_{C}^{VBF}\cdot\epsilon_{\mu\mu}} = \frac{N_{S}^{MC}[60,120]}{N_{C}^{MC}}\cdot\frac{N(Z/\gamma^{*}\rightarrow\mu\mu,\,m_{Z}^{Gen}>50 GeV)}{N(Z/\gamma^{*}\rightarrow\mu\mu,\,60<m_{Z}^{Gen}<120 GeV)}$
  \end{itemize}
  \end{block}
\vspace{-0.2cm}
  \begin{block}{\scriptsize Final formula}
    \footnotesize
    \centering
    $\begin{aligned}[t] N_{S}^{\nu\nu\,Data}=&\frac{N_{C}^{Data}-N_{C}^{BKG}}{N_{C}^{MC}(Z^{Gen}\rightarrow\mu\mu)}\cdot N_{S}^{MC}[60,120](Z^{Gen}\rightarrow\mu\mu) \\ & x \frac{N(Z/\gamma^{*}\rightarrow\mu\mu,m_{Z}^{Gen}>50 GeV)}{N(Z/\gamma^{*}\rightarrow\mu\mu,60<m_{Z}^{Gen}<120 GeV)} \cdot R[50,\infty]\end{aligned}$
    \scriptsize
    \begin{itemize}
    \item This is the same as the formula derived above
    \end{itemize}
  \end{block}
  \end{columns}
\end{frame}

\begin{frame}
  \frametitle{Preliminary Results - No Systematics}
  \begin{block}{Components}
    \scriptsize
    \begin{itemize}
    \item $N_{C}^{Data}: 12 \pm 3.4641 (stat)$ 
    \item $N_{C}^{Bkg}: 0.225755 \pm 0.118559 (stat)$
    \item $N_{S}^{MC}[60,120]: 40.9211 \pm 1.86195 (stat)$
    \item $N_{C}^{MC}(Z^{Gen}\rightarrow\mu\mu): 26.4646 \pm 1.46436 (stat)$
    \item $N(Z/\gamma^{*}\rightarrow\mu\mu,m_{Z}^{Gen}>50 GeV): 3.3216e+07 \pm 5763.33 (stat)$
    \item $N(Z/\gamma^{*}\rightarrow\mu\mu,60<m_{Z}^{Gen}<120 GeV): 3.1915e+07 \pm 5649.33 (stat)$
    \end{itemize}
  \end{block}
  \begin{block}{Preliminary Result}
    \scriptsize
    \begin{itemize}
    \item Analysis B - $98.4151 \pm 28.41 (stat.) \pm 13.6326 (MC stat.)$
    \item Analysis A - $103.452 \pm 30.4366 (stat.) \pm 14.2746 (MC stat.)$
    \end{itemize}
  \end{block}
\end{frame}

\begin{frame}
  \frametitle{What about QCD/EWK cross-section ratio difference}
  \begin{itemize}
  \item For QCD the ratio of $\nu\nu$ and $\mu\mu$ cross-sections is $~5.6$
  \item For EWK Sasha has calculated that it is ~1.6
  \item $N_{S}^{MC}$ is made up of 23 QCD events and 18 EWK events
  \item To my mind this means that the background estimation decreases considerable as a result of the much lower ratio for $~40\%$ of the signal events
  \end{itemize}
\end{frame}


\begin{frame}
  \frametitle{Conclusions}
  \label{lastframe}
  \begin{block}{}
    \scriptsize
    \begin{itemize}
    \item Method does seem consistent
    \item Is there a reason not to calculate the cross-section ratio with the mass window? 
    \item[-] It would remove the need for the additional event ratio.
    \item Preliminary results from analalysis B are compatible at at least the same level as the W estimates
    \end{itemize}
  \end{block}
\end{frame}

\begin{frame}
  \frametitle{Backup}
\end{frame}


\end{fmffile}
\end{document}
