\documentclass[hyperref=colorlinks]{beamer}
\mode<presentation>
\usetheme{iclpt}
\setbeamertemplate{navigation symbols}{}
\setbeamertemplate{headline}{
  \begin{beamercolorbox}[leftskip=.2cm,rightskip=.2cm,topskip=.2cm,ht=1.1cm,dp=0.1cm,wd=\textwidth]{institute in head/foot}
    \includegraphics[height=1cm]{icl.pdf}
    \hfill
    \includegraphics[height=1cm]{../Pics/CMS-Color.pdf}
  \end{beamercolorbox}
}
\setbeamertemplate{footline}{
  \begin{beamercolorbox}[ht=.35cm,dp=0.2cm,wd=\textwidth,leftskip=.3cm]{author in head/foot}%
    \begin{minipage}[c]{5cm}%
      \usebeamerfont{author in head/foot}
      \insertshortauthor 
      \insertshorttitle
    \end{minipage}\hfill%
    \hfill
    \insertframenumber{} / \pageref{lastframe}
    %\hfill
    \begin{minipage}{6cm}
      \hfill
      %\insertshorttitle
    \end{minipage}
  \end{beamercolorbox}%
}

\usepackage{color}
\usepackage{tabularx,colortbl}
\usepackage{graphicx}
\usepackage{pdfpages}
\usepackage{feynmp}
\usepackage{rotating}
\usepackage{moresize}
\usepackage{xcolor,colortbl}
\DeclareGraphicsRule{*}{mps}{*}{}

\title[Invisible Higgs at CMS]{\vspace{-0.2cm} Searches for invisible decay modes of the Higgs boson with the CMS detector}
\author[P. Dunne]{\underline{P. Dunne} - Imperial College London} % A.M. Magnan and A. Nikitenko Joao Pela with \\ R. Aggleton, J. Brooke: Bristol \\ C.Asawangtrakuldee, Q.Li: Peking \\ P. Srimanobhas: Chulalongkorn \\ S. Kumar, K. Mazumdar: Mumbai}
\titlegraphic{
  \vspace{-0.7cm}
  \includegraphics[width=\textwidth]{TalkPics/invcomb021213/feyndiags}
  %% \begin{fmfgraph*}(100,70)
  %%         \fmfleft{i1,i2}
  %%         \fmfright{o1,o2,o3}
  %%         \fmf{fermion}{i1,v1,o1}
  %%         \fmf{fermion}{i2,v2,o3}
  %%         \fmf{phantom,tension=4/5}{v1,v2}
  %%         \fmffreeze
  %%         \fmf{photon,label=$W,,Z$}{v1,v3}
  %%         \fmf{photon,label=$W,,Z$}{v2,v3}
  %%         \fmf{dashes}{v3,o2}
  %%         \fmflabel{$q$}{i1}
  %%         \fmflabel{$q$}{i2}
  %%         \fmflabel{$q$}{o1}
  %%         \fmflabel{$q$}{o3}
  %%         \fmflabel{$H$}{o2}

  %%       \end{fmfgraph*}
}
\date{}
\begin{document}
\begin{fmffile}{studseminarfeynmandiags}

  %TITLE PAGE
  \section{Title}
  \begin{frame}
    \titlepage
    
  \end{frame}

  \begin{frame}
    \frametitle{CMS and the LHC}
    \begin{center}
      \includegraphics[width=.9\textwidth]{TalkPics/cern-lhc-aerial.jpg}
      \end{center}
  \end{frame}

  \begin{frame}
    \frametitle{CMS}
    \includegraphics[width=\textwidth]{TalkPics/CMS_Slice.png}
  \end{frame}
  
  %HIGGS BACKGROUND
  \begin{frame}
    \frametitle{The Higgs Boson}
    \begin{columns}
      \column{.5\textwidth}
      \includegraphics[width=\textwidth]{../../Reports/Firstyearreport/XSBR_8TeV_SM.pdf}


      \column{.5\textwidth}
      \includegraphics[width=\textwidth]{TalkPics/hgg.png}      

    \end{columns}
  \end{frame}

  \begin{frame}
    \frametitle{Why Higgs to Invisible?}
    \centering
    \includegraphics[height=.85\textheight]{TalkPics/EPJCAug2014Cover.png}
  \end{frame}

  \begin{frame}
    \frametitle{Why Higgs to Invisible?}
    \vspace{-.2cm}
    \begin{columns}
      \column{.5\textwidth}
      \begin{block}{\scriptsize Experimental motivation}
        \scriptsize
        \begin{itemize}
        \item Current measurements of the 125 GeV Higgs boson are compatible with Standard Model (SM) expectations
        \item[-] large uncertainties can still accommodate significant beyond the SM (BSM) properties
        \item Additional Higgs bosons with exotic decays are not excluded
        \end{itemize}
      \end{block}
      \column{.45\textwidth}
      \hfill\includegraphics[height=.55\textheight]{TalkPics/panicpics/indirectbrbsm.pdf}
      \column{.05\textwidth}
      \begin{turn}{-90}\scriptsize CMS-PAS-HIG-14-009\end{turn}
    \end{columns}
    \begin{columns}
      \column{1.095\textwidth}
      \begin{block}{\scriptsize Theoretical motivation}
        \scriptsize
        \begin{itemize}
        \item Many BSM theories predict Higgs boson decays to invisible final states:
        \item[-] e.g. SUSY, extra dimensions, fourth-generation neutrinos
        \item These final state particles are often dark matter candidates
        \end{itemize}
      \end{block}
    \end{columns}

  \end{frame}

  \begin{frame}
    \frametitle{Direct and Indirect Searches}

    \begin{columns}
      \column{.5\textwidth}
      \vspace{-.8cm}
      \begin{block}{\scriptsize Indirect searches}
        \begin{itemize}
          \scriptsize
        \item BSM Higgs decays affect the total Higgs width:
        \item Visible decays can, therefore, constrain the invisible branching fraction
        \end{itemize}
      \end{block}
      \column{.45\textwidth}

      \hfill\includegraphics[height=.5\textheight]{TalkPics/panicpics/indirectbrbsm.pdf}
      \column{.05\textwidth}
      \begin{turn}{-90}\scriptsize CMS-PAS-HIG-14-009\end{turn}
        

    \end{columns}
    \vspace{-.5cm}
    \begin{columns}
      \column{.5\textwidth}
      \vspace{-.3cm}
      \begin{block}{\scriptsize Direct searches}
        \scriptsize
        \begin{itemize}
        \item Direct searches must be performed in channels where the Higgs recoils against a visible system
        \item We look in the VBF (left) and ZH (right) channels
        \item[-] For ZH we study the case where the Z decays to two leptons Z($\ell\ell$)H or two b quarks Z(bb)H
        \end{itemize}
      \end{block}
      \column{.01\textwidth}
      \column{.49\textwidth}
      %vbf                                                                                                                                                     
      \begin{fmfgraph*}(60,60)
        \fmfleft{i1,i2}
        \fmfright{o1,o2,o3}
        \fmf{fermion}{i1,v1,o1}
        \fmf{fermion}{i2,v2,o3}
        \fmf{photon,label=$W,,Z$,l.side=left}{v1,v4}
        \fmf{photon}{v4,v3}
        \fmf{photon,label=$W,,Z$,l.side=right}{v2,v5}
        \fmf{photon}{v5,v3}
        \fmf{dashes}{v3,v6}
        \fmf{dashes}{v6,o2}
        \fmflabel{$q$}{i1}
        \fmflabel{$q$}{i2}
        \fmflabel{$q$}{o1}
        \fmflabel{$q$}{o3}
        \fmflabel{$H$}{o2}
      \end{fmfgraph*}
      \hspace{.4cm}
      %zh
      \begin{fmfgraph*}(60,60)
        \fmfleft{i1,i2}
        \fmfright{o1,o2}
        \fmf{fermion}{i1,v1}
        \fmf{fermion}{v1,i2}
        \fmf{photon,label=$W,,Z$}{v1,v2}
        \fmf{photon}{v2,o1}
        \fmf{dashes}{v2,o2}
        \fmflabel{$q$}{i1}
        \fmflabel{$\bar{q}$}{i2}
        \fmflabel{$W,Z$}{o1}
        \fmflabel{$H$}{o2}
      \end{fmfgraph*}
    \end{columns}
  \end{frame}

  \begin{frame}
    \frametitle{VBF outline}
    \vspace{.5cm}
    \begin{columns}
      \column{.56\textwidth}
      \vspace{-.7cm}
      \begin{block}{\scriptsize Signal Topology and Selection}
        \scriptsize
        \begin{itemize}
        \item Two high $p_{T}$ VBF jets with large rapidity separation:
        \item[-] large invariant mass and rapidity separation
        \item[-] low azimuthal angle separation
        \item Large missing transverse momentum (MET)
        \end{itemize}
      \end{block}
      \vspace{-.2cm}
      \begin{block}{\scriptsize Backgrounds and Rejection Cuts}
        \scriptsize
        \begin{itemize}
        \item $W\rightarrow \ell\nu$+jets: Veto leptons
        \item $Z\rightarrow\nu\nu$+jets: Irreducible
        \item QCD multijet events:
        \item[-]Veto events with additional jets between the two selected jets (CJV)
          \scriptsize
        \item Minor backgrounds from: top and diboson
        \end{itemize}
      \end{block}
      \column{.44\textwidth}
      \centering
      \begin{fmfgraph*}(60,50)
        \fmfleft{i1,i2}
        \fmfright{o1,o2,o3}
        \fmf{fermion}{i1,v1,o1}
        \fmf{fermion}{i2,v2,o3}
        \fmf{photon,label=$W,,Z$,l.side=left}{v1,v4}
        \fmf{photon}{v4,v3}
        \fmf{photon,label=$W,,Z$,l.side=right}{v2,v5}
        \fmf{photon}{v5,v3}
        \fmf{dashes}{v3,v6}
        \fmf{dashes}{v6,o2}
        \fmflabel{$q$}{i1}
        \fmflabel{$q$}{i2}
        \fmflabel{$q$}{o1}
        \fmflabel{$q$}{o3}
        \fmflabel{$H$}{o2}
      \end{fmfgraph*}
      \vspace{.5cm}
      \begin{columns}
        \column{.9\textwidth}
        \hfill\includegraphics[clip=true,trim=0 0 0 50,height=.55\textheight]{TalkPics/panicpics/vbfmet.pdf}
        \column{.1\textwidth}
        \begin{turn}{-90}\scriptsize arXiv:1404.1344 \end{turn}
      \end{columns}
    \end{columns}
  \end{frame}
  \begin{frame}
    \frametitle{VBF background estimation}
    \begin{columns}
      \column{.6\textwidth}
      \vspace{-.3cm}
      \begin{block}{\footnotesize Data Driven Background Estimation}
        \scriptsize
        \begin{itemize}
          \item Choose control region enriched in background
          \item Use MC signal-control ratio to go to signal region:
          $N^{signal}_{Bkg} = (N^{control}_{obs}-N^{control}_{other bkgs}) \cdot \frac{N^{signal}_{MC}}{N^{control}_{MC}}.$
          \item Z+jets: Estimate using $Z\rightarrow\mu\mu$+jets events
          \item[-] Correct for difference in cross section
          \item W+jets: Estimate using $W\rightarrow\ell\nu$+jets events
        \end{itemize}

        \centering
        \begin{tabular}{|c|c|c|c|}
          \hline
          $Z\rightarrow\mu\mu$&$W\rightarrow e\nu$&$W\rightarrow \mu\nu$&$W\rightarrow \tau\nu$\\
          \hline
          $99\pm38$&$63\pm20$&$67\pm17$&$53\pm25$\\
          \hline
        \end{tabular}
      \end{block}
      \vspace{-.3cm}
      \begin{block}{\footnotesize QCD - Use ``ABCD method'' in MET and CJV}
        \scriptsize
        \begin{itemize}
        \item $N_{QCD}=30.9\pm 1.6 (stat.) \pm 23.0 (syst.)$
        \end{itemize}
      \end{block}

      \column{.43\textwidth}
      \begin{columns}
        \vspace{-.3cm}
        \column{.95\textwidth}
        \hfill\includegraphics[clip=true,trim=0 0 0 30,height=.65\textheight]{TalkPics/panicpics/vbfzreg.pdf}
        \column{.05\textwidth}
        \hspace{-.7cm}
        \begin{turn}{-90}\scriptsize CMS-TWIKI-HIG-13-030 \end{turn}
      \end{columns}
      \footnotesize
      \centering
      \begin{tabular}{c|c|c|c}
        \multicolumn{1}{c}{CJV} & \multicolumn{1}{c}{} & \multicolumn{1}{c}{} & \multicolumn{1}{c}{} \\
        \cline{2-3}
        pass & \cellcolor{red!50} B &\cellcolor{green!50} A & \\
        \cline{2-3}
        fail & \cellcolor{red!50} D &\cellcolor{red!50} C & \\
        \cline{2-3}
        \multicolumn{1}{c}{}& \multicolumn{1}{c}{$<$130} & \multicolumn{1}{c}{$>$130} & MET
      \end{tabular}
      
    \end{columns}
    
  \end{frame}
  \begin{frame}
    \frametitle{VBF results}
          \scriptsize
          \centering
          \begin{tabular}{lc}
            \hline
            Total background & $332\pm 35 (stat.) \pm 45 (syst.)$ \\ 
            \hline
            VBF H(inv.) assuming B(H$\rightarrow$inv)=100\% &  $210 \pm 30(syst.)$ \\ 
            ggF H(inv.) assuming B(H$\rightarrow$inv)=100\%& $14 \pm 11 (syst.)$ \\
            \hline
            Observed data & 390 \\
            \hline
          \end{tabular}
\vspace{-.3cm}
    \begin{columns}
      \column{1.1\textwidth}
          \begin{block}{}
            \scriptsize
            \begin{itemize}
            \item Set limits on $\sigma$x$B(H\rightarrow inv)$
              \vspace{-.1cm}
            \item[-] Perform a single bin counting experiment using $CL_{S}$ method
            \item Assuming SM Higgs production cross-section and acceptance:
              \vspace{-.1cm}
            \item[-]  observed(expected) 95\% C.L. limit on $B(H\rightarrow inv)$ for $m_{H}$=125 GeV is 65(49)\%
            \end{itemize}
          \end{block}
    \end{columns}

    \begin{columns}
      \column{.03\textwidth}
      \column{.55\textwidth}
      \begin{columns}
        \column{.9\textwidth}
      \includegraphics[clip=true,trim=0 0 0 20,width=1.1\textwidth]{TalkPics/panicpics/vbfxslimit.pdf}
      \column{.1\textwidth}
      \hspace{-.5cm}
      \begin{turn}{-90}\scriptsize arXiv:1404.1344 \end{turn}
      \end{columns}
      \column{.55\textwidth}
      \begin{columns}
        \column{.9\textwidth}
      \includegraphics[clip=true,trim=0 0 0 20,width=1.1\textwidth]{TalkPics/panicpics/vbflimit.pdf}
      \column{.1\textwidth}
      \hspace{-.5cm}
      \begin{turn}{-90}\scriptsize arXiv:1404.1344 \end{turn}
      \end{columns}
    \end{columns}
  \end{frame}

  \begin{frame}
    \frametitle{Z($\ell\ell$)H outline}
    \begin{columns}
      \column{.56\textwidth}
      \vspace{-.5cm}
      \begin{block}{\scriptsize Signal Topology and Selection}
        \scriptsize
        \begin{itemize}
        \item Two opposite sign electrons or muons
        \item[-] Consistent with Z mass
        \item Large MET that balances leptons
        \item $\leq$1 jet, no b-tagged jets, no extra leptons
        \end{itemize}
      \end{block}
      \begin{block}{\scriptsize Backgrounds}
        \scriptsize
        \begin{itemize}
        \item ZZ($\ell\ell\nu\nu$)+jets, WZ($\ell\nu\ell\ell$)+jets: From MC
        \item WW($\ell\nu\ell\nu$)+jets, $t\bar{t}$, single top, W($\ell\nu$), QCD:
        \item[-] No real Z: use $m_{\ell\ell}$ side bands
        \item[-] $N_{\ell\ell}^{sig}=N^{sig}_{e\mu}\cdot N_{\ell\ell}^{SB}/N_{e\mu}^{SB}$
        \item Z($\ell\ell$)+jets: normalised in $\gamma$+jets region
        \end{itemize}
      \end{block}
      \column{.44\textwidth}
      \centering
      \begin{fmfgraph*}(75,40)
        \fmfleft{i1,i2}
        \fmfright{o1,o2,o3}
        \fmf{fermion}{i1,v1,i2}
        \fmf{photon,label=$Z$}{v1,v2}
        \fmf{photon}{v2,v4}
        \fmf{photon,label=$Z$,l.side=right}{v4,v3}
        \fmf{fermion}{o1,v3,o2}
        \fmf{photon}{v5,v3}
        \fmf{dashes,label=$H$,l.side=left}{v2,o3}
        \fmflabel{$\ell^{+}$}{o1}
        \fmflabel{$\ell^{-}$}{o2}
        \fmflabel{$q$}{i1}
        \fmflabel{$\bar{q}$}{i2}
      \end{fmfgraph*}
      \vspace{.4cm}
      \begin{columns}
        \column{.05\textwidth}
        \column{.9\textwidth}
        \includegraphics[clip=true,trim=0 0 0 20, width=\textwidth]{TalkPics/panicpics/zllmet.pdf}
        \column{.1\textwidth}
        \hspace{-.4cm}\begin{turn}{-90}\scriptsize arXiv:1404.1344 \end{turn}
      \end{columns}
    \end{columns}

  \end{frame}

  \begin{frame}
    \frametitle{Z($\ell\ell$)H results}
    \vspace{-.2cm}
      \vspace{-.2cm}
     \begin{block}{}
       \scriptsize
       \begin{itemize}
       \item Limits obtained from a 2D fit to $m_{T}$ and $\Delta\phi (\ell\ell)$
       \item[-] 1D fit to $m_{T}$ for 7 TeV data
       \item Assuming SM Higgs production cross-section and acceptance:
       \item[-]  observed(expected) 95\% C.L. limit on $B(H\rightarrow inv)$ for $m_{H}$=125 GeV is 83(86)\%
       \end{itemize}

    \end{block}

      \begin{columns}
     \column{.5\textwidth}
       \centering
       \scriptsize arXiv:1404.1344
       \includegraphics[clip=true,trim=25 0 0 20, height=.6\textheight]{TalkPics/panicpics/zllmt.pdf}
    \column{.5\textwidth}
       \centering
       \scriptsize arXiv:1404.1344
       \includegraphics[clip=true,trim=25 0 0 20, height=.6\textheight]{TalkPics/panicpics/zlldphi.pdf}
    \end{columns}
    \end{frame}


  \begin{frame}
    \frametitle{Z(bb)H outline and backgrounds}
    \vspace{.4cm}
    \begin{columns}
      \column{.56\textwidth}
      \vspace{-.82cm}
      \begin{block}{\scriptsize Signal Topology and Selection}
        \scriptsize
        \begin{itemize}
          \vspace{-.05cm}
        \item Two b-tagged jets: high invariant mass
        \item Three categories in MET
        \end{itemize}
      \end{block}
      \vspace{-.15cm}
      \begin{block}{\scriptsize Backgrounds and Rejection Cuts}
        \scriptsize
        \begin{itemize}
          \vspace{-.05cm}
        \item Z($\nu\nu$)+jets, W($\ell\nu$)+jets
          \vspace{-.05cm}
        \item ZZ($\nu\nu b\bar{b}$)
          \vspace{-.05cm}
        \item WZ($\ell\nu b\bar{b}$), $t\bar{t}$, single top
          \vspace{-.05cm}
        \item[-] Veto events with leptons
          \scriptsize
          \vspace{-.05cm}
        \item QCD
          \vspace{-.05cm}
        \item[-] MET quality requirements
          \vspace{-.05cm}
        \end{itemize}
      \end{block}
      \vspace{-.15cm}
      \begin{block}{\scriptsize Background estimation - data normalised MC}
        \scriptsize
        \begin{itemize}
          \vspace{-.05cm}
        \item Use MC normalised with a simultaneous fit in seven control regions
        %\item[-] Z+jets (0,1,2 b-jets), W+jets (0,1,2 b-jets), $t\bar{t}$
          \vspace{-.05cm}
          
        \end{itemize}
      \end{block}
      \column{.44\textwidth}
      \centering
      \begin{fmfgraph*}(75,40)
        \fmfleft{i1,i2}
        \fmfright{o1,o2,o3}
        \fmf{fermion}{i1,v1,i2}
        \fmf{photon,label=$Z$}{v1,v2}
        \fmf{photon}{v2,v4}
        \fmf{photon,label=$Z$,l.side=right}{v4,v3}
        \fmf{fermion}{o1,v3,o2}
        \fmf{photon}{v5,v3}
        \fmf{dashes,label=$H$,l.side=left}{v2,o3}
        \fmflabel{$\bar{b}$}{o1}
        \fmflabel{$b$}{o2}
        \fmflabel{$q$}{i1}
        \fmflabel{$\bar{q}$}{i2}
      \end{fmfgraph*}
      \vspace{.45cm}
      \begin{columns}
        \column{.05\textwidth}
        \column{.9\textwidth}
        \includegraphics[clip=true,trim=0 0 0 20, width=.95\textwidth]{TalkPics/panicpics/zbbcsv.pdf}
        \column{.1\textwidth}
        \hspace{-.4cm}\begin{turn}{-90}\scriptsize arXiv:1404.1344 \end{turn}
      \end{columns}
    \end{columns}


  \end{frame}

  \begin{frame}
    \frametitle{Z($b\bar{b}$)H results}
    \vspace{-.2cm}
    \begin{columns}
      \column{.7\textwidth}
    \begin{block}{}
      \centering
      \tiny
      \begin{tabular}{lccc}
        \hline
        Process & High MET & Intermediate MET & Low MET \\
        \hline
        Total backgrounds & $181.3\pm 9.8$ & $64.8\pm 4.1$ & $40.5\pm 4.1$ \\
        Z($b\bar{b}$)H(inv) & $12.6\pm 1.1$ & $3.6\pm 0.3$ & $1.6\pm 0.1$ \\
        Observed data & 204 & 61 & 38 \\
        \hline
      \end{tabular}
    \end{block}
    \end{columns}
    \begin{columns}
      \column{.5\textwidth}
    \begin{block}{}
      \scriptsize
      \begin{itemize}
      \item Multivariate analysis (BDT):
      \item[-] performed for each mass hypothesis and boost region
        \scriptsize
      \item Limits from a fit to the BDT output distribution
       \item Assuming SM Higgs production cross-section and acceptance:
       \item[-]  observed(expected) 95\% C.L. limit on $B(H\rightarrow inv)$ for $m_{H}$=125 GeV is 182(199)\%
      \end{itemize}


    \end{block}
    \column{.5\textwidth}
    \begin{columns}
      \column{.95\textwidth}
      \vspace{.05cm}
      \includegraphics[clip=true,trim=0 0 20 0, width=\textwidth, height=.7\textheight]{TalkPics/panicpics/zbbbdt.pdf}
      \column{.05\textwidth}
              \hspace{-.2cm}\begin{turn}{-90}\scriptsize arXiv:1404.1344 \end{turn}
    \end{columns}
    \end{columns}

      
  \end{frame}


  \begin{frame}
    \frametitle{Combined Results}
    \begin{block}{}
      \scriptsize
      \begin{itemize}
      \item The individual limits on $\sigma$x$B(H\rightarrow inv)$ from the three channels are combined
      \item[-] SM production cross-sections are used to interpret this as a limit on B(H$\rightarrow$inv)
      \end{itemize}
    \end{block}
    \begin{columns}
      \column{.65\textwidth}
      \centering
      \begin{columns}
        \column{.95\textwidth}
      \includegraphics[clip=true,trim=0 0 0 0, width=1.1\textwidth]{TalkPics/panicpics/combinedlimit.pdf}
        \column{.05\textwidth}
        \hspace{-.4cm}\begin{turn}{-90}\scriptsize arXiv:1404.1344 \end{turn}
      \end{columns}
      \column{.35\textwidth}
      \scriptsize
      \begin{block}{}
        Observed (expected) limits on B(H$\rightarrow$inv) at 95\% C.L. for $m_{H}$=125 GeV

        \centering
        \begin{tabular}{lc}
          \hline
          Channel & Limit/\% \\
          \hline
          VBF & 65(49) \\
          ZH($\ell\ell$+bb) & 81(83) \\
          \hline
          VBF + ZH &{\color{red} 58(44)} \\
          \hline
        \end{tabular}
      \end{block}
    \end{columns}
  \end{frame}

  \begin{frame}
    \frametitle{Signatures of Dark Matter (DM)}
    \vspace{-.2cm}
    \begin{block}{}
      \scriptsize
      \begin{itemize}
      \item If DM couples to the Higgs the following diagrams are possible
      \end{itemize}
    \end{block}
    \vspace{-.2cm}
    \begin{columns}
      \column{.35\textwidth}
      \begin{block}{\scriptsize Direct Detection - e.g. LUX}
        \vspace{.3cm}
        \begin{fmfgraph*}(100,70)
          \fmfleft{i1,i2}
          \fmfright{o1,o2}
          \fmf{fermion}{i1,v1,o1}
          \fmf{fermion}{i2,v2,o2}
          \fmf{dashes,label=$H$}{v1,v2}
          \fmffreeze
          \fmflabel{$N$}{i1}
          \fmflabel{$\chi$}{i2}
          \fmflabel{$N$}{o1}
          \fmflabel{$\chi$}{o2}
        \end{fmfgraph*}
        \vspace{.3cm}
      \end{block}

      \column{.35\textwidth}
      \begin{block}{\scriptsize Invisible Higgs - LHC}
        \vspace{.3cm}
        \begin{fmfgraph*}(100,70)
          \fmfleft{i1,i2}
          \fmfright{o1,o2}
          \fmf{fermion}{i1,v1,i2}
          \fmf{fermion}{o1,v2,o2}
          \fmf{dashes,label=$H$}{v1,v2}
          \fmffreeze
          %\fmflabel{$f/w/Z$}{i1}
          \fmflabel{$\chi$}{o1}
          %\fmflabel{$f/W/Z$}{i2}
          \fmflabel{$\chi$}{o2}
        \end{fmfgraph*}
        \vspace{.3cm}
      \end{block}
      \column{.35\textwidth}
      \begin{block}{\scriptsize Annihilation - e.g. WMAP}
        \vspace{.3cm}
        \begin{fmfgraph*}(100,70)
          \fmfleft{i1,i2}
          \fmfright{o1,o2}
          \fmf{fermion}{i1,v1,i2}
          \fmf{fermion}{o1,v2,o2}
          \fmf{dashes,label=$H$}{v1,v2}
          \fmffreeze
          %\fmflabel{$f/w/Z$}{i1}
          \fmflabel{$\chi$}{i1}
          %\fmflabel{$f/W/Z$}{i2}
          \fmflabel{$\chi$}{i2}
        \end{fmfgraph*}
        \vspace{.3cm}
      \end{block}
    \end{columns}
    \begin{block}{}
      \scriptsize
      \begin{itemize}
      \item Limits on $\mathcal{B}$(H$\rightarrow$inv) therefore constrain Higgs Portal DM models
      \item[-] These constraints are directly comparable to those from other experiments
      \end{itemize}
    \end{block}
  \end{frame}

  \begin{frame}
    \frametitle{Dark Matter Interpretation - Results}
    \scriptsize
    \vspace{-.3cm}
    \begin{block}{}
      \begin{itemize}
      \item Use an effective field theory Higgs Portal model which translates B$(H\rightarrow inv)$ into a DM-nucleon cross-section (details in backup)
      \item At 90\% C.L. the CMS limit on B(H$\rightarrow$inv) is 51\% for a 125 GeV Higgs
      \item Consider three DM spin scenarios: scalar, vector, Majorana fermion:
      \item[-] CMS limits shown in green, blue and red respectively
      \end{itemize}
    \end{block}
        \vspace{-.1cm}
    \begin{columns}
      \column{1.1\textwidth}
      \begin{columns}
 
     \column{.6\textwidth}
        \hspace{1cm}\scriptsize arXiv:1404.1344
        \includegraphics[clip=true,trim=0 0 0 0, height=.6\textheight, width=1.2\textwidth]{TalkPics/panicpics/dmlimit.pdf}    
        \column{.01\textwidth}
      \column{.27\textwidth}
      
      \begin{block}{}
          Min, lattice and max are varying values of Higgs-nucleon coupling

 (see backup)
          
      \end{block}

      \begin{block}{}
        $\mathcal{B}(H\rightarrow inv)$ gives important exclusion in the $M_{\chi}<m_{h}/2$ region
      \end{block}
    \end{columns}
      \end{columns}
  \end{frame}

  \begin{frame}%CHECK AT END
    \frametitle{Conclusions}
    \label{lastframe}
    \begin{columns}
      \column{.5\textwidth}
    \begin{block}{}
        \footnotesize
        \begin{itemize}
        \item A Direct search for Higgs to invisible decays has been performed at CMS:
        \item[-] The VBF, Z($\ell\ell$)H and Z(bb)H channels have been considered
        \item No significant excesses are seen over the background predictions
        \item The combined limit is 58(44)\% observed (expected) at 95\% C.L. for $m_{H}=125$GeV
        \item[-] It is broadly comparable with CMS indirect limits
        \item A dark matter interpretation has been presented
        \end{itemize}
      \end{block}
        \column{.5\textwidth}
          \includegraphics[clip=true,trim=0 0 0 0, width=1.2\textwidth]{TalkPics/panicpics/combinedlimit.pdf}

    \end{columns}


  \end{frame}

  \begin{frame}%MAKE SURE OK
    \frametitle{References}
    \begin{block}{}
      \begin{itemize}
      \item CMS Higgs combination - CMS-PAS-HIG-14-009
      \item CMS Higgs to Invisible paper - arXiv:1404.1344
      \item CMS TWIKI with addition Higgs to Invisible results - https://twiki.cern.ch/twiki/bin/view/CMSPublic/
        Hig13030PubTWiki
      \end{itemize}
    \end{block}
  \end{frame}

  \begin{frame}
    \frametitle{Backup}
  \end{frame}


  \begin{frame}
    \frametitle{High mass combination}
    \centering
    \vspace{-.5cm}
    \begin{block}{}
      \footnotesize
      \begin{itemize}
      \item Z($\ell\ell$)H(inv) and VBF searches both go up to at least $m_{H}$=300 GeV
      \item The same combination method as used above was used to combine these two channels between 115 and 300 GeV
      \end{itemize}
    \end{block}


    \includegraphics[clip=true,trim=0 0 0 20, width=.68\textwidth]{TalkPics/panicpics/highmasslimit.pdf}
  \end{frame}

  \begin{frame}
    \frametitle{Other direct Limits}
    \begin{block}{}
      \begin{itemize}
      \item ATLAS also produce a limit in the Z($\ell\ell$)H channel:
      \item[-] observed (expected) 75\% (62\%) at 95\% C.L.
      \end{itemize}
    \end{block}
  \end{frame}
  
  \begin{frame}
    \frametitle{DM model}
    \begin{block}{\scriptsize Formulae}
      \scriptsize
      \begin{itemize}
      \item EFT model as described in \href{http://www.sciencedirect.com/science/article/pii/S0370269312001037}{Phys.Lett. B709 (2012) 65–69}
      \item $\sigma^{SI}_{S-N} = \frac{4\Gamma_{inv}}{m_{H}^{3}v^{2}\beta}\frac{m_{N}^{4}f_{N}^{2}}{(M_{\chi}+m_{N})^{2}}$
      \item $\sigma^{SI}_{V-N} = \frac{16\Gamma_{inv}M_{\chi}^{4}}{m_{H}^{3}v^{2}\beta(m_{H}^{4}-4M_{\chi}^{2}m_{H}^{2}+12M_{\chi}^{4})}\frac{m_{N}^{4}f_{N}^{2}}{(M_{\chi}+m_{N})^{2}}$
      \item $\sigma^{SI}_{f-N} = \frac{8\Gamma_{inv}M_{\chi}^{2}}{m_{H}^{5}v^{2}\beta^{3}}\frac{m_{N}^{4}f_{N}^{2}}{(M_{\chi}+m_{N})^{2}}$
      \item[-] $m_{N}$ is the nucleon mass, 0.939 GeV
      \item[-] $f_{N}$ is the Higgs-nucleon coupling, central value 0.326, from Phys. Rev. D 81 (2010) 01453
      \item[-] Min and max values of fN from MILC collaboration Phys. Rev. Lett. 103 (2009) 122002
      \item[-] v is the Higgs vacuum expectation, 174 GeV
      \item[-] $\beta=\sqrt{1-4M_{\chi}^{2}/m_{H}^{2}}$
      \item[-] $B(H\rightarrow inv.)=\Gamma_{inv}/(\Gamma_{SM}+\Gamma_{inv})$
      \end{itemize}
    \end{block}

  \end{frame}

\end{fmffile}
\end{document}

