\documentclass[hyperref=colorlinks]{beamer}
\mode<presentation>
\usetheme{iclpt}
\setbeamertemplate{navigation symbols}{}
\setbeamertemplate{headline}{
\begin{beamercolorbox}[leftskip=.2cm,rightskip=.2cm,topskip=.2cm,ht=1.1cm,dp=0.1cm,wd=\textwidth]{institute in head/foot}
  \includegraphics[height=1cm]{icl.pdf}
  \hfill
  \includegraphics[height=1cm]{../Pics/CMS-Color.pdf}
\end{beamercolorbox}
}
\setbeamertemplate{footline}{
\begin{beamercolorbox}[ht=.55cm,dp=0.4cm,wd=\textwidth,leftskip=.3cm]{author in head/foot}%
  \begin{minipage}[c]{5cm}%
    \usebeamerfont{author in head/foot}
    \insertshortauthor 
    \insertshorttitle
    \end{minipage}\hfill%
  \insertframenumber{} / \pageref{lastframe}
  \hfill
  \begin{minipage}{6cm}
    \hfill
  \end{minipage}
\end{beamercolorbox}%
}

\usepackage{color}
\usepackage{tabularx,colortbl}
\usepackage{graphicx}
\usepackage{pdfpages}
\usepackage{feynmp}
\DeclareGraphicsRule{*}{mps}{*}{}

\title{\vspace{-0.2cm} Combination of Higgs to Invisible Direct Measurements}
%\subtitle{AN-12-403,PAS-HIG-13-013 \vspace{-0.7cm}}
\author[P. Dunne]{\underline{P. Dunne} \\ on behalf of the H$\rightarrow$invisible analysis groups} % A.M. Magnan and A. Nikitenko Joao Pela with \\ R. Aggleton, J. Brooke: Bristol \\ C.Asawangtrakuldee, Q.Li: Peking \\ P. Srimanobhas: Chulalongkorn \\ S. Kumar, K. Mazumdar: Mumbai}
\titlegraphic{
  \vspace{-0.7cm}
%% \begin{fmfgraph*}(100,70)
%%         \fmfleft{i1,i2}
%%         \fmfright{o1,o2,o3}
%%         \fmf{fermion}{i1,v1,o1}
%%         \fmf{fermion}{i2,v2,o3}
%%         \fmf{phantom,tension=4/5}{v1,v2}
%%         \fmffreeze
%%         \fmf{photon,label=$W,,Z$}{v1,v3}
%%         \fmf{photon,label=$W,,Z$}{v2,v3}
%%         \fmf{dashes}{v3,o2}
%%         \fmflabel{$q$}{i1}
%%         \fmflabel{$q$}{i2}
%%         \fmflabel{$q$}{o1}
%%         \fmflabel{$q$}{o3}
%%         \fmflabel{$H$}{o2}
%%       \end{fmfgraph*}
}
\date{}
\begin{document}
\begin{fmffile}{feynmandiags}

%TITLE PAGE
\section{Title}
\begin{frame}
  \titlepage
  
\end{frame}

%OUTLINE
\begin{frame}
  \frametitle{Introduction}
  \begin{itemize}
  \item All three currently approved Higgs to invisible results have been combined
  \item[-] VBF (HIG-13-013), ZH$\rightarrow$ll+inv (HIG-13-018), ZH$\rightarrow$bb+inv (HIG-13-028)
  \item Updates to combination since twiki result:
  \item[-] ZH$\rightarrow$bb+inv has been included
  \item[-] Correlations between uncertainties in the three channels are now properly taken into account
  \item[-] A combination of the ZH $\rightarrow$ll and VBF channels has been performed up to 300 GeV
  \end{itemize}
\end{frame}

\begin{frame}
  \frametitle{Current Indirect Result}
  \centering
  \includegraphics[height=.6\textheight]{indirectbrbsm.png}
  \begin{itemize}
  \item observed (expected) limit of 64\% (67\%) at 95\% C.L. on $BR_{inv} $ for a 125 GeV Higgs
  \item[-] Combination between direct and indirect methods is being investigated e.g. \href{https://indico.cern.ch/getFile.py/access?contribId=3&sessionId=9&resId=1&materialId=slides&confId=267834}{talk by M. Zanetti}
  \end{itemize}
\end{frame}

\begin{frame}
  \frametitle{Datacards}
  \begin{itemize}
  \item ZH$\rightarrow$ll analysis has datacards for 105, 115, 125, 135, 145, 175, 200 \& 300 GeV
  \item ZH$\rightarrow$bb analysis has datacards for 105, 115, 125, 135, 145 \& 150 GeV
  \item VBF analysis has datacards for 110, 125, 150, 200, 300 and 400 GeV
  \item[-] New VBF datacards were produced for 115,135 and 145 GeV, with the same method as used for the twiki plot
  \end{itemize}
\end{frame}  

\begin{frame}
  \frametitle{Combination Method}
  \begin{itemize}
  \item The cards for the two approved analyses were combined using the standard Higgs combination tool
  \item[-] A bug was found in the tool that meant that lnN correlated uncertainties were not being properly treated, fixed in latest combine version
  \item Correlations between analyses were taken into account according to combination group recommendations
  \item All other uncertainties were considered fully uncorrelated between analyses
  \end{itemize}
\end{frame}
    
\begin{frame}
  \frametitle{Separate results: Direct}
  \centering
  \begin{columns}
    \column{.5\textwidth}
    \begin{itemize}
    \item VBF
    \end{itemize}
    \column{.5\textwidth}
    \begin{itemize}
    \item ZH
    \end{itemize}
  \end{columns}
  \begin{columns}
    \column{.5\textwidth}
    \includegraphics[width=\textwidth]{TalkPics/invcomb021213/vbflimit.pdf}
    \column{.5\textwidth}
    \includegraphics[width=\textwidth]{TalkPics/invcomb021213/zhlimit.pdf}
  \end{columns}
  \begin{columns}
    \column{.5\textwidth}
    \begin{itemize}
    \item observed (expected) limit of 67\% (52\%) at 95\% C.L. on $BR_{inv}$ for a 125 GeV Higgs
    \end{itemize}
    \column{.5\textwidth}
    \begin{itemize}
    \item observed (expected) limit of 81\% (83\%) at 95\% C.L. on $BR_{inv}$ for a 125 GeV Higgs
    \end{itemize}
  \end{columns}
\end{frame}

\begin{frame}
  \frametitle{Separate results: Cross-Section limits}
  \centering
  \begin{columns}
    \column{.5\textwidth}
    \begin{itemize}
    \item VBF
    \end{itemize}
    \column{.5\textwidth}
    \begin{itemize}
    \item ZH
    \end{itemize}
  \end{columns}
  \begin{columns}
    \column{.5\textwidth}
    \includegraphics[width=\textwidth]{TalkPics/invcomb021213/vbfxslimit.pdf}
    \column{.5\textwidth}
    \includegraphics[width=\textwidth]{TalkPics/invcomb021213/zhxslimit.pdf}
  \end{columns}
  \begin{columns}
    \column{.5\textwidth}
    \begin{itemize}
    \item observed (expected) limit of 67\% (52\%) at 95\% C.L. on $BR_{inv}$ for a 125 GeV Higgs
    \end{itemize}
    \column{.5\textwidth}
    \begin{itemize}
    \item observed (expected) limit of 81\% (83\%) at 95\% C.L. on $BR_{inv}$ for a 125 GeV Higgs
    \end{itemize}
  \end{columns}
\end{frame}

\begin{frame}
  \frametitle{Combined Results}
  \centering
  \vspace{-.2cm}
  \includegraphics[clip=true,trim=0 5 0 20, width=.8\textwidth]{TalkPics/invcomb021213/combinedlimit.pdf}
  \vspace{-.3cm}
  \begin{itemize}
  \item Observed (expected) limit at 125 GeV is 58(46)\%
  \end{itemize}
\end{frame}

\begin{frame}
  \frametitle{High mass combination}
  \centering
  \vspace{-.3cm}
  \begin{itemize}
  \item Z$\rightarrow$ll+inv and VBF both have datacards up to 300 GeV
  \item The same combination method as used above was used to combine these two channels between 115 and 300 GeV
  \end{itemize}
  \includegraphics[clip=true,trim=0 5 0 20, width=.75\textwidth]{TalkPics/invcomb021213/highmasslimit.pdf}
 \end{frame}


\begin{frame}
  \frametitle{Conclusions}
  \label{lastframe}
  \begin{itemize}
  \item All three Higgs to invisible channels have been combined using the standard Higgs combination tool
  \item The result is compatible with the SM at between the 1 \& 2$\sigma$ level depending on Higgs mass
  \item The combined result gives strongest direct limit on the invisible branching fraction of the SM Higgs
  \end{itemize}
\end{frame}

\begin{frame}
  \frametitle{Backup}
\end{frame}

\begin{frame}
  \frametitle{Previous Limits}
  \begin{itemize}
  \item CMS PAS limits on $BR_{inv}$ for a 125 GeV Higgs boson are:
  \item[-] VBF: observed (expected) limit of 69\% (53\%) at 95\% C.L.
  \item[-] ZH$\rightarrow$ll+inv: observed (expected) limit of 75\% (91\%) at 95\% C.L.
  \item[-] ZH$\rightarrow$bb+inv: ovserved (expected) limit of 182\% (199\%) at 95\% C.L.
  \item[-] CMS indirect limit, from visible channels: observed (expected) limit of 64\% (67\%) at 95\% C.L.
  \item ATLAS also produce an indirect limit and a limit in the ZH channel:
  \item[-] Indirect limit 60\% (no expected limit given)
  \item[-] ZH: observed (expected) 65\% (84\%)    
  \end{itemize}
\end{frame}

\begin{frame}
  \frametitle{VBF Cross-sections}
  \centering
  \begin{tabular}{|l|c|}
  \hline  
  Mass/GeV & $\sigma/pb$ \\
  \hline  
  110 & $1.809 \pm 0.048$\\
  115 & $1.729 \pm 0.046$\\
  125 & $1.578 \pm 0.042$\\
  135 & $1.448 \pm 0.038$\\
  145 & $1.333 \pm 0.035$\\
  150 & $1.280 \pm 0.033$\\
  200 & $0.869 \pm 0.023$\\
  300 & $0.441 \pm 0.011$\\
  400 & $0.254 \pm 0.007$\\
  \hline  
  \end{tabular}
\end{frame}

\begin{frame}
  \frametitle{Signal Yield interpolation}
  \begin{columns}
    \column{.5\textwidth}
    \begin{itemize}
    \item $N_{Signal}=eff. \times acc. \times \mathcal L\sigma$
    \item Luminosity is constant
    \item Yield over cross-section is thus proportional to efficiency times acceptance
    \item Signal yields were produced at 115, 125(to cross-check), 135 and 145 GeV for the VBF channel
    \item[-] Cross-sections from LHC-HXSWG were used
    \end{itemize}
    \column{.5\textwidth}
    \centering
    \hspace{-.5cm}
    \includegraphics[clip=true,trim=0 0 0 30, width=1.2\textwidth]{yieldoverxs.pdf}
  \end{columns}
\end{frame}


{
\setbeamercolor{background canvas}{bg=}
\includepdf[pages=31]{TalkPics/VBF-H-Invisible-Approval.pdf}
}

\end{fmffile}
\end{document}

